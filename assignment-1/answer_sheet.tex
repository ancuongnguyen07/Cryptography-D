\documentclass{article}      	% Style of the document                     
\usepackage{fullpage}
\usepackage{amsmath}     	   	% Maths                                          
\usepackage[utf8]{inputenc}	% UTF-8 characters                                               
\usepackage[T1]{fontenc}    	% Tuki ääkkösille (Finnish names don't cause problems)                                            
\usepackage{parskip}        		% Linebreak between paragraphs                
\usepackage{graphicx}       		% Graphics package for adding figures                        
\usepackage{epstopdf}       		% Possibility to add *.eps figures
 \usepackage{ dsfont }            % Symbol for real numbers
\usepackage{ amsfonts }
\usepackage{hyperref}
\usepackage{amsthm}
\usepackage{enumitem}        % possibility to label list items by alphabet

\usepackage[probability,adversary,sets,notions,operators,complexity,keys,primitives,asymptotics,advantage]{cryptocode} % for writing cryptography, you need to have the file 'cryptocode.sty' in the same folder as this file

\newcommand{\xor}{\, \texttt{XOR} \,} % shorthand for typing the XOR operator in mathmode

% feel free to add packages you need

\author{Cuong Nguyen 101559968}
\title{Exercise Sheet 1}

\begin{document}         
\maketitle

\section{Exercise 1: PRGs can leak half their input}
Assume that \(g_f\) is not a PRG, so there is an efficient Advantage such that
\(Adv^{\mathsf{PRG}}_{g, \adv}(\lambda)\) is non-negligible. In other words,
\(\left| \prob{\mathsf{Exp}_{g, \adv}^{\mathsf{PRG,0}}(1^\lambda) = 1} -
\prob{\mathsf{Exp}_{\adv}^{\mathsf{PRG,1}}(1^\lambda) = 1} \right|\) is non-negligible.

We have an adversary against \(f\):
\begin{center}
    \procedure{$\rdv(1^\lambda, y)$}{
        x\sample\bin^\lambda\\
        y\gets f(x_l)\\
        y'\gets y || x_r\\
        b^* \sample \adv(1^\lambda, y')\\
        \pcreturn b^*
    }
\end{center}

Then ($u$ is an uniformly random string ${0,1}^\lambda$):
\begin{align*}
    Adv^{\mathsf{PRG}}_{f,\rdv}(\lambda) &= \left| \prob{\mathsf{Exp}^{PRG, 0}_{f,\rdv} = 1} 
                                            - \prob{\mathsf{Exp}^{PRG, 1}_{\rdv} = 1}\right|\\
    &= \left| \prob{\adv(1^\lambda, y') = 1} - \prob{\adv(1^\lambda, u)} \right|\\
    &= \left| \prob{\mathsf{Exp}^{PRG, 0}_{g,\adv}(1^\lambda) = 1} - \prob{\mathsf{Exp}^{PRG, 1}_{\adv} = 1} \right|\\
    &= Adv^{\mathsf{PRG}}_{g, \adv}(\lambda)
\end{align*}

We observe that \(Adv^{\mathsf{PRG}}_{f,\rdv}(\lambda) = Adv^{\mathsf{PRG}}_{g, \adv}(\lambda)\) which is
non-negligible, leading to the contradiction of \(f\) is a PRG. Hence, if \(f\) is a PRG, then \(g_f\) is
a PRG.
\section{Some OWFs are not PRGs}
Note that \(|x|=\lambda\).
Assume that we have a length-preserving OWF \(g\). According to the appending zeros theorem,
\(h:=g(x)||0^\lambda\) is a length-expanding OWF (the stretch function \(s(\lambda)
= \lambda\) as \(|h(x)|=\lambda + s(\lambda) = 2\lambda\)).

Does security experiments on PRG \(h\):
\begin{center}
    \begin{pchstack}
      \procedure{${\mathsf{Exp}}_{h,s,\adv}^{\mathsf{PRG},0}(1^\lambda)$}{
      x\sample\bin^\lambda\\
      y\gets h(x)\\
      b^*\sample\adv(1^\lambda,y)\\
      \pcreturn b^*}
    \pchspace
      \procedure{${\mathsf{Exp}}_{s,\adv}^{\mathsf{PRG},1}(1^\lambda)$}{
      \\
      y\sample\bin^{\lambda+s(\lambda)}\\
      b^*\sample\adv(1^\lambda,y)\\
      \pcreturn b^*}
    \end{pchstack}
\end{center}

We observe that the left-half of the output of OWF \(h(x)\) is a fixed zeros string.
In other words, the expanded part of \(h\) on the uniformly random string \(x\) does
not look random at all. Hence, we can conclude that:
\begin{center}
    \[ \mathsf{Adv}^{\mathsf{PRG}}_{g,s,\adv}(\lambda):=\big|\prob{{\mathsf{Exp}}_{g,s,\adv}^{\mathsf{PRG},0}(1^\lambda) = 1} 
   - \prob{{\mathsf{Exp}}_{s,\adv}^{\mathsf{PRG},1}(1^\lambda) = 1}\big| \]
\end{center}
is non-negligible, leading to the conclusion that \(h\) is not a PRG.
% \section{PRGs are OWFs}
% Assume that \(g\) is not a OWF, so there is an efficient adversary against
% \(g\) such that \(Adv^{\mathsf{OWF}}_{g, \adv}(\lambda)\) is non-negligible.


\include{tex/exercise-4}
\include{tex/exercise-5}

\section*{Exercise 1} 

Define the function $f$ as
\begin{align*}
    f:\, &\bin^* \rightarrow \bin^*\\
    & x \mapsto x\xor 1^{\abs{x}}
\end{align*}
and show that for the following attacker $\adv$, it holds that
$\prob{\mathsf{Exp}_{f, \adv}^{\mathsf{OW}}(1^\lambda) = 1} = 1$.
Above $\xor$ means bitwise XOR operation and 

\begin{center}
    \begin{pchstack}
    \procedure{ $\adv(y, 1^{\abs{x}})$ }{ 
        z \leftarrow y \xor 1^{\abs{x}} \\
        \pcreturn z
    }
    \end{pchstack}
\end{center}

and the experiment is defined as in the lecture:
\begin{center}
    \procedure{$\mathsf{Exp}_{f, \adv}^{\mathsf{OW}}(1^\lambda)$}{
        x\sample\bin^\lambda\\
        y\gets f(x)\\
        x'\sample\adv(y, 1^\lambda)\\
        \pcif f(x') = y \pcthen\\
        \pcind \pcreturn 1\\ % \pcind indents the line
        \pcreturn 0
    }
\end{center}

\section*{Exercise 2}

\begin{align*}
    h(b||x) &=
        \begin{cases}
            0 || f(x)  & \text{if } b = 0\\
            1 || x     & \text{if } b = 1.
        \end{cases}
\end{align*}

\section*{Exercise 6}

\begin{enumerate}[label = (\alph*)] % label \alph numbers the list items by alphabet (a,b,c)
    \item The sum of two negligible functions is negligible.
    \item Multiplying a negligible function by an (arbitrary) polynomial yields a negligible function.
    \item (Very challenging) There exists a sequence of negligible functions 
    $\nu_\lambda: \NN \rightarrow[0,1]$ 
    such that the function
    $\mu(\lambda) := \sum_{i = 1}^\lambda \nu_i(\lambda)$
    is the constant $1$ function, i.e., for all $\lambda \in \NN$, it holds that $\mu(\lambda) = 1$. Hint: Use a diagonalization argument.
\end{enumerate}


\end{document}






















begin{align*}