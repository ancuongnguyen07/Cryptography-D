\section{Exercise 1: Probability and Pseudocode}

%%% Section A
\subsection*{(a) Probability for \(D_1 + D_2 + D_3 = 17 \)}
The combination that satisifies the sum of 17:
\begin{align*}
    \{6, 6, 5\}
\end{align*}
with the permutation for three dices then there are \(3!\) combinations.

In conclusion, the probability that the sum \(D_1 + D_2 + D_3 \) is equal
to 17:
\begin{align*}
    \probsub{D_1,D_2,D_3}{D_1+D_2+D_3=17} &= \frac{3!}{216} \\
    &= \frac{6}{216} \\
    &= \frac{1}{36}
\end{align*}


%%% Section B
\subsection*{(b) Shows the function \(f\) and attacker \(\adv\) that holds
            \(\prob{\mathsf{Exp}_{f, \adv}^{\mathsf{OW}}(1^\lambda) = 1} = 1\)}

The experiment \(\prob{\mathsf{Exp}_{f, \adv}^{\mathsf{OW}}(1^\lambda)}\) is defined as:
\begin{center}
    \procedure{$\mathsf{Exp}_{f, \adv}^{\mathsf{OW}}(1^\lambda)$}{
        x\sample\bin^\lambda\\
        y\gets f(x)\\
        x'\sample\adv(y, 1^\lambda)\\
        \pcif \lvert x' \rvert \neq \lambda \pcthen\\
        \pcind \pcreturn 0\\
        \pcif f(x') = y \pcthen\\
        \pcind \pcreturn 1\\ % \pcind indents the line
        \pcreturn 0
    }
\end{center}

Then we replace \(f(x)\) by \(\left(x \oplus 1^{\lvert x \rvert}\right)\) and
\(\adv(y, 1^{\lambda})\) by \(\left(y \oplus 1^{\lambda}\right)\):
\begin{center}
    \procedure{$\mathsf{Exp}_{f, \adv}^{\mathsf{OW}}(1^\lambda)$}{
        x\sample\bin^\lambda\\
        y\gets x \oplus 1^\lambda\\
        x'\sample y \oplus 1^\lambda\\
        \pcif \lvert x' \rvert \neq \lambda \pcthen\\
        \pcind \pcreturn 0\\
        \pcif f(x') = y \pcthen\\
        \pcind \pcreturn 1\\ % \pcind indents the line
        \pcreturn 0
    }
\end{center}

We notice that:
\begin{align*}
    x'\sample y \oplus 1^\lambda &\Leftrightarrow x'\sample x \oplus 1^\lambda \oplus 1^\lambda\\
    &\Leftrightarrow x'\sample x \oplus 0^\lambda\\
    &\Leftrightarrow x'\sample x
\end{align*}

With given function \(f\) and the attacker \(\adv\),
\(x'\sample x\) holds with all \(x\sample \bin^\lambda\). From that,
we infer that \(f(x') = f(x)\) or in other words \(f(x') = y\). Hence,
\(\mathsf{Exp}_{f, \adv}^{\mathsf{OW}}(1^\lambda)\) return 1 with all
\(\bin^\lambda\). In conclusion,
\(\prob{\mathsf{Exp}_{f, \adv}^{\mathsf{OW}}(1^\lambda) = 1} = 1\) holds.