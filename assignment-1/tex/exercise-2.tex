\section{Exercise 2: One-Way Functions}
%%% Statement a)
\subsubsection{(a) For all length-preserving one-way functions \(f\) and \(g\), the
following function \(h\) is a one-way function: \(h(x):=f(x)||g(x)\).}

\(h\) is a one-way function: \textbf{TRUE}.

Justification: Assuming that \(h\) is NOT a OWF, we then can infer that both
\(f\) and \(g\) are not OWF as \(h(x):= f(x)||g(x)\). Hence, if \(f\) and \(g\)
are OWFs, then \(h\) must be a OWF too. \(h\) function here is a concatenation of
the output of two OWFs.

%%% Statement b)
\subsubsection{(b) For all one-way functions \(f\) and all polynomially computable fucntions
\(b\) with one bit output, the follwing function \(h\) is a one-way function:
\(h(x):=f(x)||g(x)\).}

\(h(x)\) is a one-way function: \textbf{TRUE}.

Justification: \(h(x):=f(x)||b(x)\) where \(b(x) \rightarrow {0,1}\). \(h\) is a concatenation of
the output of OWF \(f\) and \(k \in {0,1}\). Assuming \(h\) is NOT a OWF, We can lead to a
contradiction that \(f\) is NOT a OWF. Hence, if \(f\) is a OWF, then \(h\) must be a OWF too.

%%% Statement c)
\subsubsection{(c) For all length-preserving one-way functions \(f\) and \(g\), the follwing function
\(h\) is a one-way function: \(h(x):=g(f(x))\).}

\(h\) is a one-way function: \textbf{TRUE}.

Justification: We observe that \(h(x):=g(f(x))\) where the input of OWF \(g\) is the output
of OWF \(f\). \(h\) is a OWF.

%%% Statement d)
\subsubsection{(d) For all length-preserving one-way function \(f\), the following function \(h\)
is a one-way function: \(h(x):=f(x)_{1\cdots \lceil |x|/2 \rceil}\). I.e., \(h\) returns all bits
that \(f\) returns, except for half of the bits (rounded up).}

\(h(x)\) is a one-way function: \textbf{TRUE}.

Justification: According to Theorem 2 (Ignoring one half): If \(f\) is a OWF, then
\(g^f_{ign-1}\) is OWF as well. In this case \(h\) plays a similar role as \(g^f_{ign-1}\)
as \(h\) ignores the last half of the \(f\) output. Hence, \(h\) is a OWF.

%%% Statement e)

%%% Statement f)
\subsubsection{(f) For all length-preserving one-way functions \(f, g,\) the following function
\(h\) is a one-way function: \(h(x):=f(x) \oplus g(x)\).I.e, \(h\) is the bitwise \xor of two OWFs.}

\(h\) is a one-way function: \textbf{FALSE}.

Counterexample:\\
Let's define the adversary \(\adv\):
\begin{center}
    \procedure{$\adv(y, 1^{\lambda})$}{
        z\gets 0^\lambda\\
        \pcreturn z
    }
\end{center}

Assume that there is a OWF \(T\), we infer that \(f(x):= T(x)||0^{|x|}\) and
\(g(x):= T(x)||1^{|x|}\) are OWFs as well based on appending zeros/ones theorem.
Or in other words, the given OWFs \(f\) and \(g\) are built on top of OWF \(T\).

Let's do an experiment \(\mathsf{Exp}_{h, \adv}^{\mathsf{OW}}(1^{\lambda})\):
\begin{center}
    \procedure{$\mathsf{Exp}_{h, \adv}^{\mathsf{OW}}(1^{\lambda})$}{
        x\sample {0,1}^\lambda\\
        y\gets h(x)\\
        x'\sample \adv(y,1^\lambda)\\
        \pcif |x'| \neq \lambda \pcthen\\
        \pcind \pcreturn 0\\
        \pcif h(x') = y \pcthen\\
        \pcind \pcreturn 1\\
        \pcreturn 0
    }
\end{center}

Then we plug in the definition of \(h, f,\) and \(g\):
\begin{center}
    \procedure{$\mathsf{Exp}_{h, \adv}^{\mathsf{OW}}(1^{\lambda})$}{
        x\sample {0,1}^\lambda\\
        \colorbox{BurntOrange}{$y\gets T(x)||0^\lambda \oplus T(x)||1^\lambda$}\\
        \colorbox{BurntOrange}{$x'\sample 0^\lambda$}\\
        \pcif |x'| \neq \lambda \pcthen\\
        \pcind \pcreturn 0\\
        \colorbox{BurntOrange}{$\pcif T(x')||0^\lambda \oplus T(x')||1^\lambda = y \pcthen$}\\
        \pcind \pcreturn 1\\
        \pcreturn 0
    }
\end{center}

\begin{center}
    \procedure{$\mathsf{Exp}_{h, \adv}^{\mathsf{OW}}(1^{\lambda})$}{
        x\sample {0,1}^\lambda\\
        \colorbox{BurntOrange}{$y\gets 0^\lambda || 1^\lambda$}\\
        \colorbox{BurntOrange}{$x'\sample 0^\lambda$}\\
        \pcif |x'| \neq \lambda \pcthen\\
        \pcind \pcreturn 0\\
        \colorbox{BurntOrange}{$\pcif 0^\lambda || 1^\lambda = y \pcthen$}\\
        \pcind \pcreturn 1\\
        \pcreturn 0
    }
\end{center}

So we observe that regardless of the input \(x\), the winning probability
\(\prob{\mathsf{Exp}_{f, \adv}^{\mathsf{OW}}(1^\lambda)}\) = 1. Hence,
\(h(x):= f(x) \oplus g(x)\) is not an OWF.

%%% Statement i)
\subsubsection{(i) For all length-preserving one-way functions \(f\), the following
function \(h\) is a one-way function: \(h(x):=f(1^{|x|})\).}

\(h\) is a one-way function: \textbf{FALSE}.

Counterexample: Note that \(|x|=|y|=\lambda\)\\
We observe that only the length of input \(x\) is feed into the OWF \(f\). Thus, the
randomness of \(x\) does not really matter. Let's define an efficient adversary $\adv$:
\begin{center}
    \procedure{$\adv(y)$}{
        \pcreturn 0^{|y|}
    }
\end{center}

Then do the experiment \(\mathsf{Exp}_{h, \adv}^{\mathsf{OW}}(\lambda)\):
\begin{center}
    \procedure{$\mathsf{Exp}_{h, \adv}^{\mathsf{OW}}(\lambda)$}{
        x\sample {0,1}^\lambda\\
        y\gets h(x)\\
        x'\sample \adv(y)\\
        \pcif |x'| \neq \lambda \pcthen\\
        \pcind \pcreturn 0\\
        \pcif h(x') = y \pcthen\\
        \pcind \pcreturn 1\\
        \pcreturn 0
    }
\end{center}

Replace the definition of \(\adv\):
\begin{center}
    \procedure{$\mathsf{Exp}_{h, \adv}^{\mathsf{OW}}(\lambda)$}{
        x\sample {0,1}^\lambda\\
        \colorbox{BurntOrange}{$y\gets f(1^{|x|})$}\\
        \colorbox{BurntOrange}{Or $y\gets f(1^\lambda)$}\\
        \colorbox{BurntOrange}{$x'\sample 0^{|y|}$}\\
        \colorbox{BurntOrange}{Or $x'\sample 0^{\lambda}$}\\
        \pcif |x'| \neq \lambda \pcthen\\
        \pcind \pcreturn 0\\
        \colorbox{BurntOrange}{$\pcif h(0^\lambda) = y \pcthen$}\\
        \pcind \pcreturn 1\\
        \pcreturn 0
    }
\end{center}

Then:
\begin{center}
    \procedure{$\mathsf{Exp}_{h, \adv}^{\mathsf{OW}}(\lambda)$}{
        x\sample {0,1}^\lambda\\
        y\gets f(1^\lambda)\\
        x'\sample 0^{\lambda}\\
        \pcif |x'| \neq \lambda \pcthen\\
        \pcind \pcreturn 0\\
        \colorbox{BurntOrange}{$\pcif f(1^\lambda) = f(1^\lambda) \pcthen$}\\
        \pcind \pcreturn 1\\
        \pcreturn 0
    }
\end{center}

We observe that as long as the the \(adversary\) choose the proper length of input \(x'\)
(its randomness does not matter), then \(f(x')\) always return the same output as \(f(x)\)
does. From the perspective of the experiment above, the winning probability
\(\prob{\mathsf{Exp}_{f, \adv}^{\mathsf{OW}}(1^\lambda)}\) = 1.

